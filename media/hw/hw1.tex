\documentclass[]{article}
\usepackage{lmodern}
\usepackage{amssymb,amsmath}
\usepackage{ifxetex,ifluatex}
\usepackage{fixltx2e} % provides \textsubscript
\ifnum 0\ifxetex 1\fi\ifluatex 1\fi=0 % if pdftex
  \usepackage[T1]{fontenc}
  \usepackage[utf8]{inputenc}
\else % if luatex or xelatex
  \ifxetex
    \usepackage{mathspec}
  \else
    \usepackage{fontspec}
  \fi
  \defaultfontfeatures{Ligatures=TeX,Scale=MatchLowercase}
    \usepackage{xeCJK}
    \setCJKmainfont[]{SimSun}
\fi
% use upquote if available, for straight quotes in verbatim environments
\IfFileExists{upquote.sty}{\usepackage{upquote}}{}
% use microtype if available
\IfFileExists{microtype.sty}{%
\usepackage{microtype}
\UseMicrotypeSet[protrusion]{basicmath} % disable protrusion for tt fonts
}{}
\usepackage[margin=1in]{geometry}
\usepackage{hyperref}
\hypersetup{unicode=true,
            pdftitle={Homework 1},
            pdfauthor={Put your name and student ID here},
            pdfborder={0 0 0},
            breaklinks=true}
\urlstyle{same}  % don't use monospace font for urls
\usepackage{graphicx,grffile}
\makeatletter
\def\maxwidth{\ifdim\Gin@nat@width>\linewidth\linewidth\else\Gin@nat@width\fi}
\def\maxheight{\ifdim\Gin@nat@height>\textheight\textheight\else\Gin@nat@height\fi}
\makeatother
% Scale images if necessary, so that they will not overflow the page
% margins by default, and it is still possible to overwrite the defaults
% using explicit options in \includegraphics[width, height, ...]{}
\setkeys{Gin}{width=\maxwidth,height=\maxheight,keepaspectratio}
\IfFileExists{parskip.sty}{%
\usepackage{parskip}
}{% else
\setlength{\parindent}{0pt}
\setlength{\parskip}{6pt plus 2pt minus 1pt}
}
\setlength{\emergencystretch}{3em}  % prevent overfull lines
\providecommand{\tightlist}{%
  \setlength{\itemsep}{0pt}\setlength{\parskip}{0pt}}
\setcounter{secnumdepth}{0}
% Redefines (sub)paragraphs to behave more like sections
\ifx\paragraph\undefined\else
\let\oldparagraph\paragraph
\renewcommand{\paragraph}[1]{\oldparagraph{#1}\mbox{}}
\fi
\ifx\subparagraph\undefined\else
\let\oldsubparagraph\subparagraph
\renewcommand{\subparagraph}[1]{\oldsubparagraph{#1}\mbox{}}
\fi

%%% Use protect on footnotes to avoid problems with footnotes in titles
\let\rmarkdownfootnote\footnote%
\def\footnote{\protect\rmarkdownfootnote}

%%% Change title format to be more compact
\usepackage{titling}

% Create subtitle command for use in maketitle
\providecommand{\subtitle}[1]{
  \posttitle{
    \begin{center}\large#1\end{center}
    }
}

\setlength{\droptitle}{-2em}

  \title{Homework 1}
    \pretitle{\vspace{\droptitle}\centering\huge}
  \posttitle{\par}
    \author{Put your name and student ID here}
    \preauthor{\centering\large\emph}
  \postauthor{\par}
      \predate{\centering\large\emph}
  \postdate{\par}
    \date{2020-08-31}


\begin{document}
\maketitle

\textbf{Q1}: The planet Tralfamadore has years with 500 days. There are
5 Tralfamadorans in the room. Write an expression for the probability
that no two of them have the same birthday.

How would you find the smallest \(n\) for which a room of \(n\)
Tralfamadorans has probability at least \(1/2\) of having two members
with the same birthday?

The above two questions really require some sort of assumption to get an
answer. In case you did not already provide one, what is the customary
assumption one uses in probability exercises?

\textbf{Q2}: Write an expression for \(\phi_X(t)\), the moment
generating function (MGF) of a random variable X. Find and interpret the
second derivative \(\phi{''}_X(0)\). If the MGF does not exist what
would we use instead?

\textbf{Q3}: If \(X\) and \(Y\) are uncorrelated random variables must
they be independent? If \(X\) and \(Y\) are independent random variables
must they be uncorrelated? Explain in both cases.

\textbf{Q4}: For events \(A\) and \(B\) define \(\mathbb{P}(A|B)\) in
terms of \(\mathbb{P}(A)\), \(\mathbb{P}(B)\), \(\mathbb{P}(A \cap B)\)
and \(\mathbb{P}(A\cup B)\). Write \(\mathbb{P}(A|B)\) as the
appropriate multiple of \(\mathbb{P}(B|A)\).

\textbf{Q5}: When is
\(\mathbb{P}(A\cap B \cap C)=\mathbb{P}(A)\mathbb{P}(B)\mathbb{P}(C)\)?
You need not describe every sufficient condition, just one really good
one.

\textbf{Q6}: State (a version of) Chebychev's inequality.

\textbf{Q7}: Write an expression for the variance of \(X + Y\). Of
course it is \[\mathrm{E}[(X+Y)^2]-(\mathrm{E}[X+Y])^2,\] but that is
not the expression I want. Your expression should involve
\(\mathrm{Var}(X)\) in a non-trivial way.

\textbf{Q8}: What is the probability density function of a normally
distributed random variable with mean \(\mu\) and variance \(\sigma^2\)?
What's the density of bivariate normal distribution with means
\(\mu_1,\mu_2\), variances \(\sigma_1^2,\sigma_2^2\), and correlation
coefficient \(\rho\)?

\textbf{Q9}: Assume that \(X,Y\sim N(0,1)\) independently. Denote
\[Z=\begin{cases}
|Y|,\ &X\ge 0,\\
-|Y|,\ &X<0,
\end{cases}\] Show that \(Z\sim N(0,1)\), but \(Y-Z\) is not normally
distributed. Is \((Y,Z)\) a bivariate normally distributed vector? Why?

\textbf{Q10}: Find the variance of a random variable X with the uniform
distribution on \([0, 1]\), either by working it out or stating it if
you remember the answer. (No looking it up on Baidu or elsewhere! Either
know it or derive it or do both just to be sure.) Using the known answer
for \(U[0, 1]\), how would you work out the variance of the uniform
distribution on \([-3, 3]\)?

\textbf{Q11}: We have some random variables \(X_1, X_2, X_3\) and so on.
Suppose that \(\mathbb{P}(X_i\le x)\to F(x)\) as \(i\to \infty\) for
some CDF \(F\). Does this mean that \(\mathrm{E}[X_i]\to \mathrm{E}[X]\)
where \(X\) is a random variable with distribution \(F\)? If it does,
either prove it, or state a well known theorem about it. If it does not,
then come up with a counterexample.

\textbf{Q12}: The random variables \(X_i\in\{0,1\}\) are independent and
identically distributed with \(\mathbb{P}(X_i = 1) = p\) and
\(\mathbb{P}(X_i = 0) = 1-p\). Their average is
\(\bar X = (1/n)\sum_{i=1}^n X_i\). What is \(\mathrm{Var}(\bar X)\)?
Find the answer using whatever combination of memory and derivation
works best for you.


\end{document}

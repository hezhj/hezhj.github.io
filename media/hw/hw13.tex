\documentclass[]{article}
\usepackage{lmodern}
\usepackage{amssymb,amsmath}
\usepackage{ifxetex,ifluatex}
\usepackage{fixltx2e} % provides \textsubscript
\ifnum 0\ifxetex 1\fi\ifluatex 1\fi=0 % if pdftex
  \usepackage[T1]{fontenc}
  \usepackage[utf8]{inputenc}
\else % if luatex or xelatex
  \ifxetex
    \usepackage{mathspec}
  \else
    \usepackage{fontspec}
  \fi
  \defaultfontfeatures{Ligatures=TeX,Scale=MatchLowercase}
    \usepackage{xeCJK}
    \setCJKmainfont[]{SimSun}
\fi
% use upquote if available, for straight quotes in verbatim environments
\IfFileExists{upquote.sty}{\usepackage{upquote}}{}
% use microtype if available
\IfFileExists{microtype.sty}{%
\usepackage{microtype}
\UseMicrotypeSet[protrusion]{basicmath} % disable protrusion for tt fonts
}{}
\usepackage[margin=1in]{geometry}
\usepackage{hyperref}
\hypersetup{unicode=true,
            pdftitle={Homework 13},
            pdfauthor={Put your name and student ID here},
            pdfborder={0 0 0},
            breaklinks=true}
\urlstyle{same}  % don't use monospace font for urls
\usepackage{longtable,booktabs}
\usepackage{graphicx,grffile}
\makeatletter
\def\maxwidth{\ifdim\Gin@nat@width>\linewidth\linewidth\else\Gin@nat@width\fi}
\def\maxheight{\ifdim\Gin@nat@height>\textheight\textheight\else\Gin@nat@height\fi}
\makeatother
% Scale images if necessary, so that they will not overflow the page
% margins by default, and it is still possible to overwrite the defaults
% using explicit options in \includegraphics[width, height, ...]{}
\setkeys{Gin}{width=\maxwidth,height=\maxheight,keepaspectratio}
\IfFileExists{parskip.sty}{%
\usepackage{parskip}
}{% else
\setlength{\parindent}{0pt}
\setlength{\parskip}{6pt plus 2pt minus 1pt}
}
\setlength{\emergencystretch}{3em}  % prevent overfull lines
\providecommand{\tightlist}{%
  \setlength{\itemsep}{0pt}\setlength{\parskip}{0pt}}
\setcounter{secnumdepth}{0}
% Redefines (sub)paragraphs to behave more like sections
\ifx\paragraph\undefined\else
\let\oldparagraph\paragraph
\renewcommand{\paragraph}[1]{\oldparagraph{#1}\mbox{}}
\fi
\ifx\subparagraph\undefined\else
\let\oldsubparagraph\subparagraph
\renewcommand{\subparagraph}[1]{\oldsubparagraph{#1}\mbox{}}
\fi

%%% Use protect on footnotes to avoid problems with footnotes in titles
\let\rmarkdownfootnote\footnote%
\def\footnote{\protect\rmarkdownfootnote}

%%% Change title format to be more compact
\usepackage{titling}

% Create subtitle command for use in maketitle
\providecommand{\subtitle}[1]{
  \posttitle{
    \begin{center}\large#1\end{center}
    }
}

\setlength{\droptitle}{-2em}

  \title{Homework 13}
    \pretitle{\vspace{\droptitle}\centering\huge}
  \posttitle{\par}
    \author{Put your name and student ID here}
    \preauthor{\centering\large\emph}
  \postauthor{\par}
      \predate{\centering\large\emph}
  \postdate{\par}
    \date{2021-05-27}


\begin{document}
\maketitle

\textbf{Q1}: Consider the linear model

\[y_i=\beta_0+\beta_1x_i+\epsilon_i,\ \epsilon_i\stackrel{iid}{\sim} N(0,\sigma^2), i=1,\dots,n.\]

\begin{enumerate}
\def\labelenumi{\arabic{enumi}.}
\item
  Derive the maximum likelihood estimators (MLE) for
  \(\beta_0,\beta_1\). Are they consistent with the least square
  estimators (LSE)?
\item
  Derive the MLE for \(\sigma^2\) and look at its unbiasedness. Is it
  better than the unbiased estimator \(\hat\sigma^2 = S_e^2/(n-2)\) by
  taking account for MSE?
\item
  A very slippery point is whether to treat the \(x_i\) as fixed numbers
  or as random variables. In the class, we treated the predictors
  \(x_i\) as fixed numbers for sake of convenience. Now suppose that the
  predictors \(x_i\) are iid random variables (independent of
  \(\epsilon_i\)) with density \(f_X(x;\theta)\) for some parameter
  \(\theta\). Write down the likelihood function for all of our data
  \((x_i,y_i),i=1,\dots,n\). Derive the MLE for \(\beta_0,\beta_1\) and
  see whether the MLE changes if we work with the setting of random
  predictors?
\end{enumerate}

\textbf{Q2}: Consider the linear model without intercept

\[y_i  = \beta x_i+\epsilon_i,\ i=1,\dots,n,\]

where \(\epsilon_i\) are independent with \(E[\epsilon_i]=0\) and
\(Var[\epsilon_i]=\sigma^2\).

\begin{itemize}
\item
  Write down the least square estimator \(\hat \beta\) for \(\beta\),
  and derive an unbiased estimator for \(\sigma^2\).
\item
  For fixed \(x_0\), let \(\hat{y}_0=\hat\beta x_0\). Work out
  \(Var[\hat{y}_0]\).
\end{itemize}

\textbf{Q3}: Genetic variability is thought to be a key factor in the
survival of a species, the idea being that ``diverse'' populations
should have a better chance of coping with changing environments. Table
below summarizes the results of a study designed to test that hypothesis
experimentally. Two populations of fruit flies (Drosophila serrata)-one
that was cross-bred (Strain A) and the other, in-bred (Strain B)-were
put into sealed containers where food and space were kept to a minimum.
Recorded every hundred days were the numbers of Drosophila alive in each
population.

\begin{longtable}[]{@{}llll@{}}
\toprule
Date & Day number & Strain A & Strain B\tabularnewline
\midrule
\endhead
Feb 2 & 0 & 100 & 100\tabularnewline
May 13 & 100 & 250 & 203\tabularnewline
Aug 21 & 200 & 304 & 214\tabularnewline
Nov 29 & 300 & 403 & 295\tabularnewline
Mar 8 & 400 & 446 & 330\tabularnewline
Jun 16 & 500 & 482 & 324\tabularnewline
\bottomrule
\end{longtable}

\begin{itemize}
\item
  Plot day numbers versus population sizes for Strain A and Strain B,
  respectively. Does the plot look linear? If so, please use least
  squares to figure out the coefficients and their standard errors, and
  plot the two regression lines.
\item
  Let \(\beta_1^A\) and \(\beta_1^B\) be the true slopes (i.e., growth
  rates) for Strain A and Strain B, respectively. Assume the population
  sizes for the two strains are independent. Under the same assumptions
  of \(\epsilon_i\stackrel{iid}{\sim} N(0,\sigma^2)\) for both strains,
  do we have enough evidence here to reject the null hypothesis that
  \(\beta_1^A\le \beta_1^B\) (significance level \(\alpha=0.05\))? Or
  equivalently, do these data support the theory that genetically mixed
  populations have a better chance of survival in hostile environments.
  (提示:仿照方差相同的两个正态总体均值差的假设检验,构造相应的t检验统计量)
\end{itemize}


\end{document}

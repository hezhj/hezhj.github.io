\documentclass[]{article}
\usepackage{lmodern}
\usepackage{amssymb,amsmath}
\usepackage{ifxetex,ifluatex}
\usepackage{fixltx2e} % provides \textsubscript
\ifnum 0\ifxetex 1\fi\ifluatex 1\fi=0 % if pdftex
  \usepackage[T1]{fontenc}
  \usepackage[utf8]{inputenc}
\else % if luatex or xelatex
  \ifxetex
    \usepackage{mathspec}
  \else
    \usepackage{fontspec}
  \fi
  \defaultfontfeatures{Ligatures=TeX,Scale=MatchLowercase}
    \usepackage{xeCJK}
    \setCJKmainfont[]{SimSun}
\fi
% use upquote if available, for straight quotes in verbatim environments
\IfFileExists{upquote.sty}{\usepackage{upquote}}{}
% use microtype if available
\IfFileExists{microtype.sty}{%
\usepackage{microtype}
\UseMicrotypeSet[protrusion]{basicmath} % disable protrusion for tt fonts
}{}
\usepackage[margin=1in]{geometry}
\usepackage{hyperref}
\hypersetup{unicode=true,
            pdftitle={Homework 14},
            pdfauthor={Put your name and student ID here},
            pdfborder={0 0 0},
            breaklinks=true}
\urlstyle{same}  % don't use monospace font for urls
\usepackage{graphicx,grffile}
\makeatletter
\def\maxwidth{\ifdim\Gin@nat@width>\linewidth\linewidth\else\Gin@nat@width\fi}
\def\maxheight{\ifdim\Gin@nat@height>\textheight\textheight\else\Gin@nat@height\fi}
\makeatother
% Scale images if necessary, so that they will not overflow the page
% margins by default, and it is still possible to overwrite the defaults
% using explicit options in \includegraphics[width, height, ...]{}
\setkeys{Gin}{width=\maxwidth,height=\maxheight,keepaspectratio}
\IfFileExists{parskip.sty}{%
\usepackage{parskip}
}{% else
\setlength{\parindent}{0pt}
\setlength{\parskip}{6pt plus 2pt minus 1pt}
}
\setlength{\emergencystretch}{3em}  % prevent overfull lines
\providecommand{\tightlist}{%
  \setlength{\itemsep}{0pt}\setlength{\parskip}{0pt}}
\setcounter{secnumdepth}{0}
% Redefines (sub)paragraphs to behave more like sections
\ifx\paragraph\undefined\else
\let\oldparagraph\paragraph
\renewcommand{\paragraph}[1]{\oldparagraph{#1}\mbox{}}
\fi
\ifx\subparagraph\undefined\else
\let\oldsubparagraph\subparagraph
\renewcommand{\subparagraph}[1]{\oldsubparagraph{#1}\mbox{}}
\fi

%%% Use protect on footnotes to avoid problems with footnotes in titles
\let\rmarkdownfootnote\footnote%
\def\footnote{\protect\rmarkdownfootnote}

%%% Change title format to be more compact
\usepackage{titling}

% Create subtitle command for use in maketitle
\providecommand{\subtitle}[1]{
  \posttitle{
    \begin{center}\large#1\end{center}
    }
}

\setlength{\droptitle}{-2em}

  \title{Homework 14}
    \pretitle{\vspace{\droptitle}\centering\huge}
  \posttitle{\par}
    \author{Put your name and student ID here}
    \preauthor{\centering\large\emph}
  \postauthor{\par}
      \predate{\centering\large\emph}
  \postdate{\par}
    \date{2021-06-03}


\begin{document}
\maketitle

\textbf{Q1}: Let us consider fitting a straight line,
\(y = \beta_0+\beta_1x\), to points \((x_i,y_i)\), where
\(i=1,\dots,n\).

\begin{enumerate}
\def\labelenumi{\arabic{enumi}.}
\item
  Write down the normal equations for the simple linear model via the
  matrix formalism.
\item
  Solve the normal equations by the matrix approach and see whether the
  solutions agree with the earlier calculation derived in the simple
  linear models.
\end{enumerate}

\textbf{Q2}: Consider fitting the curve \(y = \beta_0x+\beta_1x^2\) to
points (\(x_i,y_i\)), where \(i = 1,\dots,n\).

\begin{enumerate}
\def\labelenumi{\arabic{enumi}.}
\item
  Use the matrix formalism to find expressions for the least squares
  estimates of \(\beta_0\) and \(\beta_1\).
\item
  Find an expression for the covariance matrix of the estimates.
\end{enumerate}

\textbf{Q3}: Suppose that in the model

\[y_i= \beta_0+\beta_1x_i+\epsilon_i,\ i=1,\dots,n,\]

the errors \(\epsilon_i\) have mean zero and are uncorrelated, but
\(\mathrm{Var}(\epsilon_i) = \rho_i^2\sigma^2\), where the \(\rho_i>0\)
are known constants, so the errors do not have equal variance. Because
the variances are not equal, the theory developed in our class does not
apply.

\begin{enumerate}
\def\labelenumi{(\alph{enumi})}
\item
  Try to transform suitably the model such that the basic assumptions
  (i.e., the errors have zero mean and equal variance, and are
  uncorrelated) of the standard statistical model are satisfied.
\item
  Find the least squares estimates of \(\beta_0\) and \(\beta_1\) for
  the transformed model.
\item
  Find the variances of the estimates of Part (b).
\end{enumerate}

\textbf{Q4}: Consider the multiple linear model
\(Y = X\beta +\epsilon\), where \(X\) is the \(n\times p\) design
matrix, \(\beta=(\beta_0,\dots,\beta_{p-1})^\top\) is a vector of \(p\)
parameters, and the error \(\epsilon\sim N(0,\sigma^2 I_n)\). Now
consider the problem of estimating
\(\theta = \beta_0+\beta_1+\dots+\beta_{p-1}\). Assume that
\(\mathrm{rank}(X)=p<n\). Let
\(\hat\beta=(\hat\beta_0,\dots,\hat\beta_{p-1})^\top\) be the least
squares estimate of \(\beta\). Let
\(\hat\theta=\hat\beta_0+\hat\beta_1+\dots+\hat\beta_{p-1}\).

\begin{enumerate}
\def\labelenumi{(\alph{enumi})}
\item
  Show that \(\hat\theta\) is an unbaised estimate of \(\theta\).
\item
  Find the variance of the estimate \(\hat\theta\).
\item
  Let \(\hat\theta_c=c^\top Y\) be an unbiased estimate of \(\theta\)
  for any \(\beta\in \mathbb{R}^{p\times 1}\), where
  \(c\in \mathbb{R}^{n\times 1}\) is any fixed vector. Prove that
  \(\mathrm{Var}(\hat\theta_c)\ge \mathrm{Var}(\hat\theta)\). (Notice
  that \(\hat\theta\) is also a linear combination of \(y_i\). This
  result implies that \(\hat\theta\) is the best linear unbiased
  estimator for \(\theta\).)
\end{enumerate}

\textbf{Q5}: Consider the linear model in matrix formalism

\[
\boldsymbol{Y} = \boldsymbol{X}\boldsymbol {\beta} + \boldsymbol\epsilon,
\] where \(\boldsymbol Y=(y_1,\dots,y_n)^\top\),
\(\boldsymbol\beta=(\beta_0,\dots,\beta_{p-1})^\top\), \(\boldsymbol X\)
is the \(n\times p\) design matrix, and
\(\boldsymbol\epsilon=(\epsilon_1,\dots,\epsilon_n)^\top\sim N(\boldsymbol 0,\sigma^2 I_n)\)
with unknown \(\sigma>0\). Assume that
\(\mathrm{rank}(\boldsymbol{X})=r<p\).

\begin{enumerate}
\def\labelenumi{(\alph{enumi})}
\item
  Show that the least squares estimator (LSE) for \(\boldsymbol\beta\)
  is not unique.
\item
  Show that there exists an \(n\times r\) submatrix \(\boldsymbol{X}^*\)
  of \(\boldsymbol{X}\) with rank \(r\) such that
  \(\boldsymbol{X}=\boldsymbol{X}^*\boldsymbol{Q}\), where
  \(\boldsymbol{Q}\) is a \(r\times p\) matrix.
\item
  Let \(\boldsymbol\beta^* = \boldsymbol{Q\beta}\). Then the linear
  model becomes
  \(\boldsymbol{Y} = \boldsymbol{X}^*\boldsymbol {\beta}^* + \boldsymbol\epsilon\).
  Find an LSE for \(\boldsymbol\beta^*\) and show that the LSE is
  unique. Find an unbiased estimate of \(\sigma^2\) and show its
  variance.
\end{enumerate}


\end{document}

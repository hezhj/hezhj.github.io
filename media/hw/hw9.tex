\documentclass[]{article}
\usepackage{lmodern}
\usepackage{amssymb,amsmath}
\usepackage{ifxetex,ifluatex}
\usepackage{fixltx2e} % provides \textsubscript
\ifnum 0\ifxetex 1\fi\ifluatex 1\fi=0 % if pdftex
  \usepackage[T1]{fontenc}
  \usepackage[utf8]{inputenc}
\else % if luatex or xelatex
  \ifxetex
    \usepackage{mathspec}
  \else
    \usepackage{fontspec}
  \fi
  \defaultfontfeatures{Ligatures=TeX,Scale=MatchLowercase}
    \usepackage{xeCJK}
    \setCJKmainfont[]{SimSun}
\fi
% use upquote if available, for straight quotes in verbatim environments
\IfFileExists{upquote.sty}{\usepackage{upquote}}{}
% use microtype if available
\IfFileExists{microtype.sty}{%
\usepackage{microtype}
\UseMicrotypeSet[protrusion]{basicmath} % disable protrusion for tt fonts
}{}
\usepackage[margin=1in]{geometry}
\usepackage{hyperref}
\hypersetup{unicode=true,
            pdftitle={Homework 9},
            pdfauthor={Put your name and student ID here},
            pdfborder={0 0 0},
            breaklinks=true}
\urlstyle{same}  % don't use monospace font for urls
\usepackage{graphicx,grffile}
\makeatletter
\def\maxwidth{\ifdim\Gin@nat@width>\linewidth\linewidth\else\Gin@nat@width\fi}
\def\maxheight{\ifdim\Gin@nat@height>\textheight\textheight\else\Gin@nat@height\fi}
\makeatother
% Scale images if necessary, so that they will not overflow the page
% margins by default, and it is still possible to overwrite the defaults
% using explicit options in \includegraphics[width, height, ...]{}
\setkeys{Gin}{width=\maxwidth,height=\maxheight,keepaspectratio}
\IfFileExists{parskip.sty}{%
\usepackage{parskip}
}{% else
\setlength{\parindent}{0pt}
\setlength{\parskip}{6pt plus 2pt minus 1pt}
}
\setlength{\emergencystretch}{3em}  % prevent overfull lines
\providecommand{\tightlist}{%
  \setlength{\itemsep}{0pt}\setlength{\parskip}{0pt}}
\setcounter{secnumdepth}{0}
% Redefines (sub)paragraphs to behave more like sections
\ifx\paragraph\undefined\else
\let\oldparagraph\paragraph
\renewcommand{\paragraph}[1]{\oldparagraph{#1}\mbox{}}
\fi
\ifx\subparagraph\undefined\else
\let\oldsubparagraph\subparagraph
\renewcommand{\subparagraph}[1]{\oldsubparagraph{#1}\mbox{}}
\fi

%%% Use protect on footnotes to avoid problems with footnotes in titles
\let\rmarkdownfootnote\footnote%
\def\footnote{\protect\rmarkdownfootnote}

%%% Change title format to be more compact
\usepackage{titling}

% Create subtitle command for use in maketitle
\providecommand{\subtitle}[1]{
  \posttitle{
    \begin{center}\large#1\end{center}
    }
}

\setlength{\droptitle}{-2em}

  \title{Homework 9}
    \pretitle{\vspace{\droptitle}\centering\huge}
  \posttitle{\par}
    \author{Put your name and student ID here}
    \preauthor{\centering\large\emph}
  \postauthor{\par}
      \predate{\centering\large\emph}
  \postdate{\par}
    \date{2021-04-29}


\begin{document}
\maketitle

\textbf{Q1}: Let \(X_1,\dots,X_n\) be an iid sample of Possion
distribution with parameter \(\lambda>0\). Find an approximate
\(100(1-\alpha)\%\) confidence interval for \(\lambda\).

\textbf{Q2}: Suppose that an event \(A\) was observed 36 times out of
120 independent experiments. Use CLT to find an approximate \(95\%\)
confidence interval for \(P(A)\).

\textbf{Q3}: Let \(X_1,\dots,X_n\) be an iid sample from a distribution
with CDF \(F(x)\).

\begin{enumerate}
\def\labelenumi{(\alph{enumi})}
\item
  Show that the empirical CDF \(\hat F_n(x)\) is an unbiased estimate of
  \(F(x)\) for any fixed \(x\in\mathbb{R}\).
\item
  Find the variance of \(\hat F_n(x)\).
\item
  Now suppose that \(F(x)=1-\exp(-\lambda x)\) for \(x>0\) and \(0\)
  otherwise. Inspecting whether the variance of \(\hat F_n(x)\) attains
  the lower bound of Cramer-Rao inequality for estimating \(F(x)\) with
  fixed \(x>0\). (In fact, there exists a better unbiased estimator for
  \(F(x)\) than the empirical CDF for this case.)
\end{enumerate}

\textbf{Q4}: 两个样本比较问题(开放性题目)

两个样本比较是统计中很常见的问题。如,比较男生和女生的某些指标(身高、体重、考试成绩等);在医学上,为了验证某药物的效果,需要比较实验组(服用药物)和对照组(没有服用药物)的差异性。两样本问题数不胜数。为了方便比较,我们经常做出这样的假设:两个样本所属的总体为独立的正态总体。在课上,我们讨论了两个独立正态总体均值差异和方差差异,给出相应的点估计和区间估计。你的任务如下:

\begin{enumerate}
\def\labelenumi{\arabic{enumi}.}
\tightlist
\item
  根据感兴趣的应用背景,自选两样本数据。这些数据可以通过发放调查问卷得到,也可以使用已有的数据。如果是通过调查问卷获得,需要说明问卷的内容和调查的对象等信息。如果是使用已有的数据,需要引用数据的来源以及说明该数据的相关背景。
\item
  描述性数据分析:通过图表等方式比较两样本数据,比如箱线图、密度直方图/核估计等
\item
  异常数据处理(如有)
\item
  正态性假设的验证:通过观察密度估计图像,判断是否能够用正态假设进行分析
\item
  分别求出每个总体的均值和方差的点估计和95\%置信区间(需要说明你所用到的假设)
\item
  分别求出均值差和方差比的点估计和95\%置信区间(需要说明你所用到的假设)
\item
  基于上述分析得出你的结论:回答这两个样本是否存在差异,并由此能得到什么有价值的信息。比如,你正在比较实验组(服用药物)和对照组(没有服用药物)的差异性,如果发现这两个样本没有差异,则或许可以说明该药物对治疗没有效果。
\end{enumerate}

PS: 如果实在找不到合适的数据,可以使用R package \texttt{dslabs}
中的身高数据``heights'' (in inches).


\end{document}

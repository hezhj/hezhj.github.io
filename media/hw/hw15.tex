\documentclass[]{article}
\usepackage{lmodern}
\usepackage{amssymb,amsmath}
\usepackage{ifxetex,ifluatex}
\usepackage{fixltx2e} % provides \textsubscript
\ifnum 0\ifxetex 1\fi\ifluatex 1\fi=0 % if pdftex
  \usepackage[T1]{fontenc}
  \usepackage[utf8]{inputenc}
\else % if luatex or xelatex
  \ifxetex
    \usepackage{mathspec}
  \else
    \usepackage{fontspec}
  \fi
  \defaultfontfeatures{Ligatures=TeX,Scale=MatchLowercase}
    \usepackage{xeCJK}
    \setCJKmainfont[]{SimSun}
\fi
% use upquote if available, for straight quotes in verbatim environments
\IfFileExists{upquote.sty}{\usepackage{upquote}}{}
% use microtype if available
\IfFileExists{microtype.sty}{%
\usepackage{microtype}
\UseMicrotypeSet[protrusion]{basicmath} % disable protrusion for tt fonts
}{}
\usepackage[margin=1in]{geometry}
\usepackage{hyperref}
\hypersetup{unicode=true,
            pdftitle={Homework 15},
            pdfauthor={Put your name and student ID here},
            pdfborder={0 0 0},
            breaklinks=true}
\urlstyle{same}  % don't use monospace font for urls
\usepackage{color}
\usepackage{fancyvrb}
\newcommand{\VerbBar}{|}
\newcommand{\VERB}{\Verb[commandchars=\\\{\}]}
\DefineVerbatimEnvironment{Highlighting}{Verbatim}{commandchars=\\\{\}}
% Add ',fontsize=\small' for more characters per line
\usepackage{framed}
\definecolor{shadecolor}{RGB}{248,248,248}
\newenvironment{Shaded}{\begin{snugshade}}{\end{snugshade}}
\newcommand{\AlertTok}[1]{\textcolor[rgb]{0.94,0.16,0.16}{#1}}
\newcommand{\AnnotationTok}[1]{\textcolor[rgb]{0.56,0.35,0.01}{\textbf{\textit{#1}}}}
\newcommand{\AttributeTok}[1]{\textcolor[rgb]{0.77,0.63,0.00}{#1}}
\newcommand{\BaseNTok}[1]{\textcolor[rgb]{0.00,0.00,0.81}{#1}}
\newcommand{\BuiltInTok}[1]{#1}
\newcommand{\CharTok}[1]{\textcolor[rgb]{0.31,0.60,0.02}{#1}}
\newcommand{\CommentTok}[1]{\textcolor[rgb]{0.56,0.35,0.01}{\textit{#1}}}
\newcommand{\CommentVarTok}[1]{\textcolor[rgb]{0.56,0.35,0.01}{\textbf{\textit{#1}}}}
\newcommand{\ConstantTok}[1]{\textcolor[rgb]{0.00,0.00,0.00}{#1}}
\newcommand{\ControlFlowTok}[1]{\textcolor[rgb]{0.13,0.29,0.53}{\textbf{#1}}}
\newcommand{\DataTypeTok}[1]{\textcolor[rgb]{0.13,0.29,0.53}{#1}}
\newcommand{\DecValTok}[1]{\textcolor[rgb]{0.00,0.00,0.81}{#1}}
\newcommand{\DocumentationTok}[1]{\textcolor[rgb]{0.56,0.35,0.01}{\textbf{\textit{#1}}}}
\newcommand{\ErrorTok}[1]{\textcolor[rgb]{0.64,0.00,0.00}{\textbf{#1}}}
\newcommand{\ExtensionTok}[1]{#1}
\newcommand{\FloatTok}[1]{\textcolor[rgb]{0.00,0.00,0.81}{#1}}
\newcommand{\FunctionTok}[1]{\textcolor[rgb]{0.00,0.00,0.00}{#1}}
\newcommand{\ImportTok}[1]{#1}
\newcommand{\InformationTok}[1]{\textcolor[rgb]{0.56,0.35,0.01}{\textbf{\textit{#1}}}}
\newcommand{\KeywordTok}[1]{\textcolor[rgb]{0.13,0.29,0.53}{\textbf{#1}}}
\newcommand{\NormalTok}[1]{#1}
\newcommand{\OperatorTok}[1]{\textcolor[rgb]{0.81,0.36,0.00}{\textbf{#1}}}
\newcommand{\OtherTok}[1]{\textcolor[rgb]{0.56,0.35,0.01}{#1}}
\newcommand{\PreprocessorTok}[1]{\textcolor[rgb]{0.56,0.35,0.01}{\textit{#1}}}
\newcommand{\RegionMarkerTok}[1]{#1}
\newcommand{\SpecialCharTok}[1]{\textcolor[rgb]{0.00,0.00,0.00}{#1}}
\newcommand{\SpecialStringTok}[1]{\textcolor[rgb]{0.31,0.60,0.02}{#1}}
\newcommand{\StringTok}[1]{\textcolor[rgb]{0.31,0.60,0.02}{#1}}
\newcommand{\VariableTok}[1]{\textcolor[rgb]{0.00,0.00,0.00}{#1}}
\newcommand{\VerbatimStringTok}[1]{\textcolor[rgb]{0.31,0.60,0.02}{#1}}
\newcommand{\WarningTok}[1]{\textcolor[rgb]{0.56,0.35,0.01}{\textbf{\textit{#1}}}}
\usepackage{longtable,booktabs}
\usepackage{graphicx,grffile}
\makeatletter
\def\maxwidth{\ifdim\Gin@nat@width>\linewidth\linewidth\else\Gin@nat@width\fi}
\def\maxheight{\ifdim\Gin@nat@height>\textheight\textheight\else\Gin@nat@height\fi}
\makeatother
% Scale images if necessary, so that they will not overflow the page
% margins by default, and it is still possible to overwrite the defaults
% using explicit options in \includegraphics[width, height, ...]{}
\setkeys{Gin}{width=\maxwidth,height=\maxheight,keepaspectratio}
\IfFileExists{parskip.sty}{%
\usepackage{parskip}
}{% else
\setlength{\parindent}{0pt}
\setlength{\parskip}{6pt plus 2pt minus 1pt}
}
\setlength{\emergencystretch}{3em}  % prevent overfull lines
\providecommand{\tightlist}{%
  \setlength{\itemsep}{0pt}\setlength{\parskip}{0pt}}
\setcounter{secnumdepth}{0}
% Redefines (sub)paragraphs to behave more like sections
\ifx\paragraph\undefined\else
\let\oldparagraph\paragraph
\renewcommand{\paragraph}[1]{\oldparagraph{#1}\mbox{}}
\fi
\ifx\subparagraph\undefined\else
\let\oldsubparagraph\subparagraph
\renewcommand{\subparagraph}[1]{\oldsubparagraph{#1}\mbox{}}
\fi

%%% Use protect on footnotes to avoid problems with footnotes in titles
\let\rmarkdownfootnote\footnote%
\def\footnote{\protect\rmarkdownfootnote}

%%% Change title format to be more compact
\usepackage{titling}

% Create subtitle command for use in maketitle
\providecommand{\subtitle}[1]{
  \posttitle{
    \begin{center}\large#1\end{center}
    }
}

\setlength{\droptitle}{-2em}

  \title{Homework 15}
    \pretitle{\vspace{\droptitle}\centering\huge}
  \posttitle{\par}
    \author{Put your name and student ID here}
    \preauthor{\centering\large\emph}
  \postauthor{\par}
      \predate{\centering\large\emph}
  \postdate{\par}
    \date{2021-06-10}


\begin{document}
\maketitle

\textbf{Q1}: Consider the multiple linear regression model \[
\boldsymbol{Y} = \boldsymbol{X}\boldsymbol {\beta} + \boldsymbol\epsilon,
\] where \(\boldsymbol Y=(y_1,\dots,y_n)^\top\),
\(\boldsymbol\beta=(\beta_0,\dots,\beta_{p-1})^\top\), \(\boldsymbol X\)
is the \(n\times p\) design matrix, and
\(\boldsymbol\epsilon=(\epsilon_1,\dots,\epsilon_n)^\top\). Assume that
\(\mathrm{rank}(X)=p<n\), \(E[\boldsymbol\epsilon]=\boldsymbol 0\), and
\(\mathrm{Var}[\boldsymbol\epsilon]= \sigma^2 I_n\) with \(\sigma>0\).

(a). Show that the covariance matrix of the least squares estimates is
diagonal if and only if the columns of \(\boldsymbol{X}\),
\(\boldsymbol{X}_1,\dots,\boldsymbol{X}_p\), are orthogonal, that is
\(\boldsymbol{X}_i^\top \boldsymbol{X}_j=0\) for \(i\neq j\).

(b). Let \(\hat y_i\) and \(\hat\epsilon_i\) be the fitted values and
the residuals, respectively. Show that
\(n\sigma^2 = \sum_{i=1}^n \mathrm{Var}[\hat y_i]+\sum_{i=1}^n\mathrm{Var}[\hat\epsilon_i]\).

(c). Suppose further that
\(\boldsymbol\epsilon\sim N(\boldsymbol 0,\sigma^2 I_n)\), and you use F
test to handle the hypothesis
\[H_0: \beta_1=\beta_2=\dots=\beta_{p-1}=0\ vs.\ H_1:\sum_{i=1}^{p-1} \beta_i^2\neq0.\]
If the coefficient of determination \(R^2=0.58\), \(p = 5\) and
\(n=15\), is the null rejected at the significance level
\(\alpha =0.05\)?
(\(F_{0.95}(4,10)=3.48,F_{0.95}(5,10)=3.33,t_{0.95}(10)=1.81\))

\textbf{Q2}: Consider the simple linear model

\[y_i= \beta_0+\beta_1x_i+\epsilon_i,\ \epsilon_i\stackrel{iid}{\sim} N(0,\sigma^2).\]

Use the F-test method derived in the multiple linear model to test the
hypothesis \(H_0:\beta_1=0\ vs.\ H_1:\beta_1\neq 0\), and see whether
the F-test agrees with the earlier t-test derived in the simple linear
models.

\textbf{Q3}:
这是一份1994年收集1379个对象关于收入(earn)、身高(height,单位:英寸)、性别(sex)、教育水平(ed,单位:年)、年龄(age)等信息的数据集。下面展示\textbf{前6组数据(不是完整数据)}。

根据完整的数据集:
\url{http://www.hezhijian.com.cn/media/hw/wages.csv},建立回归模型探索收入与其他变量之间的关系.
根据回归的结果回答以下问题:

\begin{enumerate}
\def\labelenumi{\arabic{enumi}.}
\tightlist
\item
  长的越高的人挣钱越多?男性是否比女性挣得多?是否年龄越大挣得越多?高学历是不是挣得更多?
\item
  请根据你自己的个人数据预测下你在未来不同年龄段收入的置信区间和预测区间。PS:
  这里身高单位为英寸(1厘米=0.3937008英寸),教育年长是指你一共读了多少年的书(不算幼儿园时间,例如大部分大三学生读书时间=\(6+3+3+3=15\),
  当然留级或者跳级除外.)。
\end{enumerate}

如果使用R,下面这种方式可以得到该数据集。

\begin{Shaded}
\begin{Highlighting}[]
\NormalTok{data=}\KeywordTok{read.csv}\NormalTok{(}\StringTok{'http://www.hezhijian.com.cn//media//hw//wages.csv'}\NormalTok{)}
\NormalTok{knitr}\OperatorTok{::}\KeywordTok{kable}\NormalTok{(}\KeywordTok{head}\NormalTok{(data),}\DataTypeTok{caption =} \StringTok{'收入数据集前六行'}\NormalTok{)}
\end{Highlighting}
\end{Shaded}

\begin{longtable}[]{@{}rrlrr@{}}
\caption{收入数据集前六行}\tabularnewline
\toprule
earn & height & sex & ed & age\tabularnewline
\midrule
\endfirsthead
\toprule
earn & height & sex & ed & age\tabularnewline
\midrule
\endhead
79571.30 & 73.89 & male & 16 & 49\tabularnewline
96396.99 & 66.23 & female & 16 & 62\tabularnewline
48710.67 & 63.77 & female & 16 & 33\tabularnewline
80478.10 & 63.22 & female & 16 & 95\tabularnewline
82089.35 & 63.08 & female & 17 & 43\tabularnewline
15313.35 & 64.53 & female & 15 & 30\tabularnewline
\bottomrule
\end{longtable}

\textbf{Q4}:
Pima数据集来自美国国立糖尿病、消化与肾脏疾病研究所,用于研究比马(Pima)印第安妇女糖尿病情况。该数据集中的自变量为
怀孕次数(npreg)、血糖浓度(glu)、血压(bp)、皮肤厚度(skin)、体重指数(bmi)、糖尿病谱
系功能(ped)、年龄(age),因变量(type)为该妇女是否患有糖尿病。下面展示\textbf{前6组数据(不是完整数据)}。根据完整的数据集:
\url{http://www.hezhijian.com.cn/media/hw/pima.csv},分别用Logit和Probit二值回归模型分析糖尿病患病率与其他变量之间的关系.

如果使用R,下面这种方式可以得到Pima数据集。

\begin{Shaded}
\begin{Highlighting}[]
\KeywordTok{library}\NormalTok{(MASS)}
\NormalTok{pima =}\StringTok{ }\KeywordTok{rbind}\NormalTok{(Pima.tr,Pima.te) }\CommentTok{##完整数据集}
\NormalTok{knitr}\OperatorTok{::}\KeywordTok{kable}\NormalTok{(}\KeywordTok{head}\NormalTok{(pima),}\DataTypeTok{caption =} \StringTok{'Pima数据集前六行'}\NormalTok{)}
\end{Highlighting}
\end{Shaded}

\begin{longtable}[]{@{}rrrrrrrl@{}}
\caption{Pima数据集前六行}\tabularnewline
\toprule
npreg & glu & bp & skin & bmi & ped & age & type\tabularnewline
\midrule
\endfirsthead
\toprule
npreg & glu & bp & skin & bmi & ped & age & type\tabularnewline
\midrule
\endhead
5 & 86 & 68 & 28 & 30.2 & 0.364 & 24 & No\tabularnewline
7 & 195 & 70 & 33 & 25.1 & 0.163 & 55 & Yes\tabularnewline
5 & 77 & 82 & 41 & 35.8 & 0.156 & 35 & No\tabularnewline
0 & 165 & 76 & 43 & 47.9 & 0.259 & 26 & No\tabularnewline
0 & 107 & 60 & 25 & 26.4 & 0.133 & 23 & No\tabularnewline
5 & 97 & 76 & 27 & 35.6 & 0.378 & 52 & Yes\tabularnewline
\bottomrule
\end{longtable}


\end{document}

\documentclass[]{article}
\usepackage{lmodern}
\usepackage{amssymb,amsmath}
\usepackage{ifxetex,ifluatex}
\usepackage{fixltx2e} % provides \textsubscript
\ifnum 0\ifxetex 1\fi\ifluatex 1\fi=0 % if pdftex
  \usepackage[T1]{fontenc}
  \usepackage[utf8]{inputenc}
\else % if luatex or xelatex
  \ifxetex
    \usepackage{mathspec}
  \else
    \usepackage{fontspec}
  \fi
  \defaultfontfeatures{Ligatures=TeX,Scale=MatchLowercase}
    \usepackage{xeCJK}
    \setCJKmainfont[]{SimSun}
\fi
% use upquote if available, for straight quotes in verbatim environments
\IfFileExists{upquote.sty}{\usepackage{upquote}}{}
% use microtype if available
\IfFileExists{microtype.sty}{%
\usepackage{microtype}
\UseMicrotypeSet[protrusion]{basicmath} % disable protrusion for tt fonts
}{}
\usepackage[margin=1in]{geometry}
\usepackage{hyperref}
\hypersetup{unicode=true,
            pdftitle={Homework 6},
            pdfauthor={Put your name and student ID here},
            pdfborder={0 0 0},
            breaklinks=true}
\urlstyle{same}  % don't use monospace font for urls
\usepackage{graphicx,grffile}
\makeatletter
\def\maxwidth{\ifdim\Gin@nat@width>\linewidth\linewidth\else\Gin@nat@width\fi}
\def\maxheight{\ifdim\Gin@nat@height>\textheight\textheight\else\Gin@nat@height\fi}
\makeatother
% Scale images if necessary, so that they will not overflow the page
% margins by default, and it is still possible to overwrite the defaults
% using explicit options in \includegraphics[width, height, ...]{}
\setkeys{Gin}{width=\maxwidth,height=\maxheight,keepaspectratio}
\IfFileExists{parskip.sty}{%
\usepackage{parskip}
}{% else
\setlength{\parindent}{0pt}
\setlength{\parskip}{6pt plus 2pt minus 1pt}
}
\setlength{\emergencystretch}{3em}  % prevent overfull lines
\providecommand{\tightlist}{%
  \setlength{\itemsep}{0pt}\setlength{\parskip}{0pt}}
\setcounter{secnumdepth}{0}
% Redefines (sub)paragraphs to behave more like sections
\ifx\paragraph\undefined\else
\let\oldparagraph\paragraph
\renewcommand{\paragraph}[1]{\oldparagraph{#1}\mbox{}}
\fi
\ifx\subparagraph\undefined\else
\let\oldsubparagraph\subparagraph
\renewcommand{\subparagraph}[1]{\oldsubparagraph{#1}\mbox{}}
\fi

%%% Use protect on footnotes to avoid problems with footnotes in titles
\let\rmarkdownfootnote\footnote%
\def\footnote{\protect\rmarkdownfootnote}

%%% Change title format to be more compact
\usepackage{titling}

% Create subtitle command for use in maketitle
\providecommand{\subtitle}[1]{
  \posttitle{
    \begin{center}\large#1\end{center}
    }
}

\setlength{\droptitle}{-2em}

  \title{Homework 6}
    \pretitle{\vspace{\droptitle}\centering\huge}
  \posttitle{\par}
    \author{Put your name and student ID here}
    \preauthor{\centering\large\emph}
  \postauthor{\par}
      \predate{\centering\large\emph}
  \postdate{\par}
    \date{2020-10-13}


\begin{document}
\maketitle

\hypertarget{censored-data-analysis}{%
\subsection{Censored data analysis}\label{censored-data-analysis}}

Suppose that \(Y_i\) are iid sample with PDF \(f(y;\theta)\) and CDF
\(F(y;\theta)\). But if \(Y_i> t_i\) then we don't see \(Y_i\) we only
learn that \(Y_i> t_i\). Let \(\delta_i=1\) if \(Y_i\) was observed and
\(\delta_i=0\) otherwise. The likelihood function is given by
\[L(\theta)=\prod_{i=1}^n f(y_i;\theta)^{\delta_i}(1-F(t_i;\theta))^{1-\delta_i}=\prod_{i=1}^n f(x_i;\theta)^{\delta_i}S(x_i;\theta)^{1-\delta_i},\]
where \(x_i = \min(y_i,t_i)\) denote the observed data, and
\(S(t;\theta)=1-F(t;\theta)\) is called the survival function in the
context of survival analysis.

In our class, we have derived MLE for exponential population, i.e.,
\(Y_i\stackrel{iid}\sim Exp(\lambda)\). We now consider a more flexible
distribution -- \textbf{Weibull distribution}. The Weibull distribution
with shape parameter \(\gamma>0\) and rate parameter \(\lambda>0\) has a
density given by
\[f(y;\gamma,\lambda) = \lambda\gamma y^{\gamma-1} \exp(- \lambda y^\gamma)\]
for \(y > 0\). The CDF is
\(F(y;\gamma,\lambda) = 1 - \exp(- \lambda y^\gamma)\) on \(y > 0\).
Particularly, if \(\gamma=1\), the Weibull distribution turns out to be
an Exponential distribution, and thus it is more flexible. Now suppose
\(Y_i\stackrel{iid}\sim f(y;\gamma,\lambda)\). Please answer the
following questions:

Q1. Derive MLEs for the parameters \(\gamma\) and \(\lambda\).

Q2. Show the estimated parameters for the two real datasets \texttt{aml}
and \texttt{lung} in R package \texttt{survival}, respectively. You may
use some numerical algorithm, such as Newton-Raphson algorithm.

Q3. Show the estimated average survival time and plot the estimated
survival function \(\hat S(t;\theta)\) as a function of \(t\) using the
results in Q2.

Q4. Compare the results in Q3 with the associated results for
Exponential population, which was done in our class.


\end{document}

\documentclass[]{article}
\usepackage{lmodern}
\usepackage{amssymb,amsmath}
\usepackage{ifxetex,ifluatex}
\usepackage{fixltx2e} % provides \textsubscript
\ifnum 0\ifxetex 1\fi\ifluatex 1\fi=0 % if pdftex
  \usepackage[T1]{fontenc}
  \usepackage[utf8]{inputenc}
\else % if luatex or xelatex
  \ifxetex
    \usepackage{mathspec}
  \else
    \usepackage{fontspec}
  \fi
  \defaultfontfeatures{Ligatures=TeX,Scale=MatchLowercase}
    \usepackage{xeCJK}
    \setCJKmainfont[]{SimSun}
\fi
% use upquote if available, for straight quotes in verbatim environments
\IfFileExists{upquote.sty}{\usepackage{upquote}}{}
% use microtype if available
\IfFileExists{microtype.sty}{%
\usepackage{microtype}
\UseMicrotypeSet[protrusion]{basicmath} % disable protrusion for tt fonts
}{}
\usepackage[margin=1in]{geometry}
\usepackage{hyperref}
\hypersetup{unicode=true,
            pdftitle={Homework 3},
            pdfauthor={Put your name and student ID here},
            pdfborder={0 0 0},
            breaklinks=true}
\urlstyle{same}  % don't use monospace font for urls
\usepackage{graphicx,grffile}
\makeatletter
\def\maxwidth{\ifdim\Gin@nat@width>\linewidth\linewidth\else\Gin@nat@width\fi}
\def\maxheight{\ifdim\Gin@nat@height>\textheight\textheight\else\Gin@nat@height\fi}
\makeatother
% Scale images if necessary, so that they will not overflow the page
% margins by default, and it is still possible to overwrite the defaults
% using explicit options in \includegraphics[width, height, ...]{}
\setkeys{Gin}{width=\maxwidth,height=\maxheight,keepaspectratio}
\IfFileExists{parskip.sty}{%
\usepackage{parskip}
}{% else
\setlength{\parindent}{0pt}
\setlength{\parskip}{6pt plus 2pt minus 1pt}
}
\setlength{\emergencystretch}{3em}  % prevent overfull lines
\providecommand{\tightlist}{%
  \setlength{\itemsep}{0pt}\setlength{\parskip}{0pt}}
\setcounter{secnumdepth}{0}
% Redefines (sub)paragraphs to behave more like sections
\ifx\paragraph\undefined\else
\let\oldparagraph\paragraph
\renewcommand{\paragraph}[1]{\oldparagraph{#1}\mbox{}}
\fi
\ifx\subparagraph\undefined\else
\let\oldsubparagraph\subparagraph
\renewcommand{\subparagraph}[1]{\oldsubparagraph{#1}\mbox{}}
\fi

%%% Use protect on footnotes to avoid problems with footnotes in titles
\let\rmarkdownfootnote\footnote%
\def\footnote{\protect\rmarkdownfootnote}

%%% Change title format to be more compact
\usepackage{titling}

% Create subtitle command for use in maketitle
\providecommand{\subtitle}[1]{
  \posttitle{
    \begin{center}\large#1\end{center}
    }
}

\setlength{\droptitle}{-2em}

  \title{Homework 3}
    \pretitle{\vspace{\droptitle}\centering\huge}
  \posttitle{\par}
    \author{Put your name and student ID here}
    \preauthor{\centering\large\emph}
  \postauthor{\par}
      \predate{\centering\large\emph}
  \postdate{\par}
    \date{2020-09-24}


\begin{document}
\maketitle

\textbf{Q1}: Let \(X_1,\dots,X_{15}\) be a simple random sample of
\(N(0,2^2)\). What is the distribution of
\[Y=\frac{X_1^2+\dots+X_{10}^2}{2(X_{11}^2+\dots+X_{15}^2)}?\]

\textbf{Q2}: Let \(\boldsymbol Z = \boldsymbol c + \boldsymbol{AY}\),
where \(\boldsymbol Y\) is a random vector, \(\boldsymbol A\) is a fixed
matrix, and \(\boldsymbol c\) is a fixed vector. Prove that

\begin{enumerate}
\def\labelenumi{\arabic{enumi}.}
\item
  the expected value of \(\boldsymbol Z\):
  \(\mathbb E[\boldsymbol Z]=\boldsymbol c + \boldsymbol{A}\mathbb E[\boldsymbol Y],\)
\item
  the covariance matrix of \(\boldsymbol Z\):
  \(\mathrm{Var}[\boldsymbol Z]=\boldsymbol A \mathrm{Var}[\boldsymbol Y] \boldsymbol A^\top\).
\end{enumerate}

\textbf{Q3}: Let \(X_1,\dots,X_n\) be iid sample from
\(N(\mu,\sigma^2)\), where \(\mu,\sigma\) are unknown parameters. Which
of the following are statistics? ( ) 多选题

A. \(X_1+X_n-2\mu\)

B. \((X_1-\mu)/\sigma\)

C. \((\bar X-10)/5\)

D. \(\frac 1 n\sum_{i=1}^n(X_i-S_n)^2\)

\textbf{Q4}: Suppose that
\(n=15,\bar x_{n}=168, s_n=11.43, x_{n+1}=170\). Find the values for
\(\bar x_{n+1},s_{n+1}^2\) and \(s_{n+1}^{*2}\).

\textbf{Q5}: Show that if \(X\) and \(Y\) are independent exponential
random variables with \(\lambda = 1\), then \(X/Y\) follows an F
distribution. Also, identify the degrees of freedom.

\textbf{Q6}: Suppose that
\(X=(X_1,\dots,X_n)^\top\sim N(\mathbf 0,I_n)\). Show that for any
orthogonal matrix \(U\), then \(UX\sim N(\mathbf 0,I_n)\).


\end{document}

\documentclass[]{article}
\usepackage{lmodern}
\usepackage{amssymb,amsmath}
\usepackage{ifxetex,ifluatex}
\usepackage{fixltx2e} % provides \textsubscript
\ifnum 0\ifxetex 1\fi\ifluatex 1\fi=0 % if pdftex
  \usepackage[T1]{fontenc}
  \usepackage[utf8]{inputenc}
\else % if luatex or xelatex
  \ifxetex
    \usepackage{mathspec}
  \else
    \usepackage{fontspec}
  \fi
  \defaultfontfeatures{Ligatures=TeX,Scale=MatchLowercase}
    \usepackage{xeCJK}
    \setCJKmainfont[]{SimSun}
\fi
% use upquote if available, for straight quotes in verbatim environments
\IfFileExists{upquote.sty}{\usepackage{upquote}}{}
% use microtype if available
\IfFileExists{microtype.sty}{%
\usepackage{microtype}
\UseMicrotypeSet[protrusion]{basicmath} % disable protrusion for tt fonts
}{}
\usepackage[margin=1in]{geometry}
\usepackage{hyperref}
\hypersetup{unicode=true,
            pdftitle={Homework 10},
            pdfauthor={Put your name and student ID here},
            pdfborder={0 0 0},
            breaklinks=true}
\urlstyle{same}  % don't use monospace font for urls
\usepackage{graphicx,grffile}
\makeatletter
\def\maxwidth{\ifdim\Gin@nat@width>\linewidth\linewidth\else\Gin@nat@width\fi}
\def\maxheight{\ifdim\Gin@nat@height>\textheight\textheight\else\Gin@nat@height\fi}
\makeatother
% Scale images if necessary, so that they will not overflow the page
% margins by default, and it is still possible to overwrite the defaults
% using explicit options in \includegraphics[width, height, ...]{}
\setkeys{Gin}{width=\maxwidth,height=\maxheight,keepaspectratio}
\IfFileExists{parskip.sty}{%
\usepackage{parskip}
}{% else
\setlength{\parindent}{0pt}
\setlength{\parskip}{6pt plus 2pt minus 1pt}
}
\setlength{\emergencystretch}{3em}  % prevent overfull lines
\providecommand{\tightlist}{%
  \setlength{\itemsep}{0pt}\setlength{\parskip}{0pt}}
\setcounter{secnumdepth}{0}
% Redefines (sub)paragraphs to behave more like sections
\ifx\paragraph\undefined\else
\let\oldparagraph\paragraph
\renewcommand{\paragraph}[1]{\oldparagraph{#1}\mbox{}}
\fi
\ifx\subparagraph\undefined\else
\let\oldsubparagraph\subparagraph
\renewcommand{\subparagraph}[1]{\oldsubparagraph{#1}\mbox{}}
\fi

%%% Use protect on footnotes to avoid problems with footnotes in titles
\let\rmarkdownfootnote\footnote%
\def\footnote{\protect\rmarkdownfootnote}

%%% Change title format to be more compact
\usepackage{titling}

% Create subtitle command for use in maketitle
\providecommand{\subtitle}[1]{
  \posttitle{
    \begin{center}\large#1\end{center}
    }
}

\setlength{\droptitle}{-2em}

  \title{Homework 10}
    \pretitle{\vspace{\droptitle}\centering\huge}
  \posttitle{\par}
    \author{Put your name and student ID here}
    \preauthor{\centering\large\emph}
  \postauthor{\par}
      \predate{\centering\large\emph}
  \postdate{\par}
    \date{2021-05-07}


\begin{document}
\maketitle

\textbf{Q1}: True or false, and state why:

\begin{enumerate}
\def\labelenumi{\arabic{enumi}.}
\tightlist
\item
  The significance level of a statistical test is equal to the
  probability that the null hypothesis is true.
\item
  If the significance level of a test is decreased, the power of the
  test would be expected to increase.
\item
  The probability that the null hypothesis is falsely rejected is equal
  to the power of the test.
\item
  A type I error occurs when the test statistic falls in the rejection
  region of the test.
\end{enumerate}

\textbf{Q2}: A coin is thrown independently 10 times to test the
hypothesis that the probability of heads is \(1/2\) versus the
alternative that the probability is not \(1/2\). The test rejects if
either 0 or 10 heads are observed.

\begin{enumerate}
\def\labelenumi{\arabic{enumi}.}
\tightlist
\item
  What is the significance level of the test?
\item
  If in fact the probability of heads is \(0.1\), what is the power of
  the test?
\end{enumerate}

\textbf{Q3}: Suppose that \(X_1,X_2,X_3\) are samples of Bernoulli
\(B(1,p)\) population. For testing the hypothesis
\(H_0:p=1/2\ vs.\ H_1:p=3/4\), we use a rejection region:
\[W=\{(x_1,x_2,x_3):x_1+x_2+x_3\ge 2\}.\]

\textbf{Q4}: Let \(X_1,\dots,X_n\) be an iid sample of
\(N(\mu,\sigma^2)\), where \(\mu\) is known. Show that this model has a
monotone likelihood ratio. Given a significance level \(\alpha\), derive
a UMP test of the following hypotheses:
\[H_0:\sigma^2 \ge \sigma_0^2\ vs.\ H_1:\sigma^2<\sigma_0^2;\]
\[H_0:\sigma^2 \le \sigma_0^2\ vs.\ H_1:\sigma^2>\sigma_0^2.\]

\textbf{Q5}: Let \(X_1,\dots,X_n\) be an iid sample of the double
exponential distribution with PDF
\(f(x) = \frac 12\lambda\exp(-\lambda|x|)\), where \(\lambda>0\) is
unknown. Show that this model has a monotone likelihood ratio. Given a
significance level \(\alpha\), derive a UMP test of the following
hypotheses: \[H_0:\lambda \ge \lambda_0\ vs.\ H_1:\lambda < \lambda_0;\]
\[H_0:\lambda \le \lambda_0\ vs.\ H_1:\lambda > \lambda_0.\]

\textbf{Q6}: Under the setting of Q5, derive a test (not necessarily
UMP) of the two-sided hypothesis
\[H_0:\lambda =\lambda_0\ vs.\ H_1:\lambda \neq \lambda_0\] for a given
level of significance \(\alpha\).

\textbf{Q7}: Under the setting of Q5, derive a UMP test of the
hypothesis \[H_0:\lambda > \lambda_0\ vs.\ H_1:\lambda \le \lambda_0\]
for a given level of significance \(\alpha\). (Hint: prove that the
result in Q5 is also UMP for this case)


\end{document}

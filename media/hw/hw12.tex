\documentclass[]{article}
\usepackage{lmodern}
\usepackage{amssymb,amsmath}
\usepackage{ifxetex,ifluatex}
\usepackage{fixltx2e} % provides \textsubscript
\ifnum 0\ifxetex 1\fi\ifluatex 1\fi=0 % if pdftex
  \usepackage[T1]{fontenc}
  \usepackage[utf8]{inputenc}
\else % if luatex or xelatex
  \ifxetex
    \usepackage{mathspec}
  \else
    \usepackage{fontspec}
  \fi
  \defaultfontfeatures{Ligatures=TeX,Scale=MatchLowercase}
    \usepackage{xeCJK}
    \setCJKmainfont[]{SimSun}
\fi
% use upquote if available, for straight quotes in verbatim environments
\IfFileExists{upquote.sty}{\usepackage{upquote}}{}
% use microtype if available
\IfFileExists{microtype.sty}{%
\usepackage{microtype}
\UseMicrotypeSet[protrusion]{basicmath} % disable protrusion for tt fonts
}{}
\usepackage[margin=1in]{geometry}
\usepackage{hyperref}
\hypersetup{unicode=true,
            pdftitle={Homework 12},
            pdfauthor={Put your name and student ID here},
            pdfborder={0 0 0},
            breaklinks=true}
\urlstyle{same}  % don't use monospace font for urls
\usepackage{longtable,booktabs}
\usepackage{graphicx,grffile}
\makeatletter
\def\maxwidth{\ifdim\Gin@nat@width>\linewidth\linewidth\else\Gin@nat@width\fi}
\def\maxheight{\ifdim\Gin@nat@height>\textheight\textheight\else\Gin@nat@height\fi}
\makeatother
% Scale images if necessary, so that they will not overflow the page
% margins by default, and it is still possible to overwrite the defaults
% using explicit options in \includegraphics[width, height, ...]{}
\setkeys{Gin}{width=\maxwidth,height=\maxheight,keepaspectratio}
\IfFileExists{parskip.sty}{%
\usepackage{parskip}
}{% else
\setlength{\parindent}{0pt}
\setlength{\parskip}{6pt plus 2pt minus 1pt}
}
\setlength{\emergencystretch}{3em}  % prevent overfull lines
\providecommand{\tightlist}{%
  \setlength{\itemsep}{0pt}\setlength{\parskip}{0pt}}
\setcounter{secnumdepth}{0}
% Redefines (sub)paragraphs to behave more like sections
\ifx\paragraph\undefined\else
\let\oldparagraph\paragraph
\renewcommand{\paragraph}[1]{\oldparagraph{#1}\mbox{}}
\fi
\ifx\subparagraph\undefined\else
\let\oldsubparagraph\subparagraph
\renewcommand{\subparagraph}[1]{\oldsubparagraph{#1}\mbox{}}
\fi

%%% Use protect on footnotes to avoid problems with footnotes in titles
\let\rmarkdownfootnote\footnote%
\def\footnote{\protect\rmarkdownfootnote}

%%% Change title format to be more compact
\usepackage{titling}

% Create subtitle command for use in maketitle
\providecommand{\subtitle}[1]{
  \posttitle{
    \begin{center}\large#1\end{center}
    }
}

\setlength{\droptitle}{-2em}

  \title{Homework 12}
    \pretitle{\vspace{\droptitle}\centering\huge}
  \posttitle{\par}
    \author{Put your name and student ID here}
    \preauthor{\centering\large\emph}
  \postauthor{\par}
      \predate{\centering\large\emph}
  \postdate{\par}
    \date{2020-11-24}


\begin{document}
\maketitle

\textbf{Q1}: True or false, and state why:

\begin{enumerate}
\def\labelenumi{\arabic{enumi}.}
\item
  If the p-value is 0.03, the corresponding test will reject at the
  significance level 0.02.
\item
  If a test rejects at significance level 0.06, then the p-value is less
  than or equal to 0.06.
\item
  The p-value of a test is the probability that the null hypothesis is
  correct.
\end{enumerate}

\textbf{Q2}: Mutual funds are investment vehicles consisting of a
portfolio of various types of investments. If such an investment is to
meet annual spending needs, the owner of shares in the fund is
interested in the average of the annual returns of the fund. Investors
are also concerned with the volatility of the annual returns, measured
by the variance or standard deviation. One common method of evaluating a
mutual fund is to compare it to a benchmark, the Lipper Average being
one of these. This index number is the average of returns from a
universe of mutual funds. The Global Rock Fund is a typical mutual fund,
with heavy investments in international funds. It claimed to best the
Lipper Average in terms of volatility over the period from 1989 through
2007. Its returns are given in the table below.

\begin{longtable}[]{@{}llll@{}}
\toprule
Year & Investment Return \% & Year & Investment Return \%\tabularnewline
\midrule
\endhead
1989 & 15.32 & 1999 & 27.43\tabularnewline
1990 & 1.62 & 2000 & 8.57\tabularnewline
1991 & 28.43 & 2001 & 1.88\tabularnewline
1992 & 11.91 & 2002 & −7.96\tabularnewline
1993 & 20.71 & 2003 & 35.98\tabularnewline
1994 & −2.15 & 2004 & 14.27\tabularnewline
1995 & 23.29 & 2005 & 10.33\tabularnewline
1996 & 15.96 & 2006 & 15.94\tabularnewline
1997 & 11.12 & 2007 & 16.71\tabularnewline
1998 & 0.37 & &\tabularnewline
\bottomrule
\end{longtable}

The standard deviation for the Lipper Average is \(11.67\%\). Let
\(\sigma^2\) denote the variance of the population represented by the
return percentages shown in the table above. Consider the test

\[H_0: \sigma^2\ge(11.67)^2\ vs.\ H_1:\sigma^2<(11.67)^2.\]

\begin{itemize}
\item
  If the significance level \(\alpha=0.05\), what's your decision?
\item
  Show up the p-value of your test.
\end{itemize}

\textbf{Q3}: The National Center for Health Statistics (1970) gives the
following data on distribution of suicides in the United States by month
in 1970. Is there any evidence that the suicide rate varies seasonally,
or are the data consistent with the hypothesis that the rate is constant
(the significance level \(\alpha=0.05\))? (Hint: Under the latter
hypothesis, model the number of suicides in each month as a multinomial
random variable with the appropriate probabilities and conduct a
goodness-of-fit test.)

\begin{longtable}[]{@{}lll@{}}
\toprule
Month & Number of Suicides & Days/Month\tabularnewline
\midrule
\endhead
Jan. & 1867 & 31\tabularnewline
Feb. & 1789 & 28\tabularnewline
Mar. & 1944 & 31\tabularnewline
Apr. & 2094 & 30\tabularnewline
May & 2097 & 31\tabularnewline
June & 1981 & 30\tabularnewline
July & 1887 & 31\tabularnewline
Aug. & 2024 & 31\tabularnewline
Sept. & 1928 & 30\tabularnewline
Oct. & 2032 & 31\tabularnewline
Nov. & 1978 & 30\tabularnewline
Dec. & 1859 & 31\tabularnewline
\bottomrule
\end{longtable}

\textbf{Q4}: Under (the assumption of) simple Mendelian inheritance, a
cross between plants of two particular genotypes produces progeny 1/4 of
which are ``dwarf'' and \(3/4\) of which are ``giant'', respectively. In
an experiment to determine if this assumption is reasonable, a cross
results in progeny having 243 dwarf and 682 giant plants. If ``giant''
is taken as success, the null hypothesis is that \(p =3/4\) and the
alternative that \(p \neq 3/4\).

\begin{itemize}
\item
  Let \(X_i,i=1,\dots,n\) be the sample of the population \(B(1,p)\). By
  central limit theorem (CLT), the distribution of \(\bar X\) can be
  approximated by a normal distribution \(N(p,p(1-p)/n)\). Please use
  this approximation to do the binominal test above.
\item
  Actually, we can do the exact binominal test according to the formula
  given in P.114 of our Chinese textbook. Compare the results in the
  exact test and the approximate test for significance levels
  \(\alpha=0.05,0.01,0.001\).
\end{itemize}


\end{document}
